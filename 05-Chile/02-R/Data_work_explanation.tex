% Options for packages loaded elsewhere
\PassOptionsToPackage{unicode}{hyperref}
\PassOptionsToPackage{hyphens}{url}
%
\documentclass[
]{article}
\usepackage{amsmath,amssymb}
\usepackage{lmodern}
\usepackage{iftex}
\ifPDFTeX
  \usepackage[T1]{fontenc}
  \usepackage[utf8]{inputenc}
  \usepackage{textcomp} % provide euro and other symbols
\else % if luatex or xetex
  \usepackage{unicode-math}
  \defaultfontfeatures{Scale=MatchLowercase}
  \defaultfontfeatures[\rmfamily]{Ligatures=TeX,Scale=1}
\fi
% Use upquote if available, for straight quotes in verbatim environments
\IfFileExists{upquote.sty}{\usepackage{upquote}}{}
\IfFileExists{microtype.sty}{% use microtype if available
  \usepackage[]{microtype}
  \UseMicrotypeSet[protrusion]{basicmath} % disable protrusion for tt fonts
}{}
\makeatletter
\@ifundefined{KOMAClassName}{% if non-KOMA class
  \IfFileExists{parskip.sty}{%
    \usepackage{parskip}
  }{% else
    \setlength{\parindent}{0pt}
    \setlength{\parskip}{6pt plus 2pt minus 1pt}}
}{% if KOMA class
  \KOMAoptions{parskip=half}}
\makeatother
\usepackage{xcolor}
\usepackage[left=2cm,right=2cm,top=2cm,bottom=2cm]{geometry}
\usepackage{graphicx}
\makeatletter
\def\maxwidth{\ifdim\Gin@nat@width>\linewidth\linewidth\else\Gin@nat@width\fi}
\def\maxheight{\ifdim\Gin@nat@height>\textheight\textheight\else\Gin@nat@height\fi}
\makeatother
% Scale images if necessary, so that they will not overflow the page
% margins by default, and it is still possible to overwrite the defaults
% using explicit options in \includegraphics[width, height, ...]{}
\setkeys{Gin}{width=\maxwidth,height=\maxheight,keepaspectratio}
% Set default figure placement to htbp
\makeatletter
\def\fps@figure{htbp}
\makeatother
\setlength{\emergencystretch}{3em} % prevent overfull lines
\providecommand{\tightlist}{%
  \setlength{\itemsep}{0pt}\setlength{\parskip}{0pt}}
\setcounter{secnumdepth}{5}
\usepackage{booktabs}
\usepackage{rotating, graphicx}
\usepackage{dcolumn}
\usepackage{float}
\usepackage{enumitem}
\usepackage{booktabs}
\usepackage{longtable}
\usepackage{array}
\usepackage{multirow}
\usepackage{wrapfig}
\usepackage{float}
\usepackage{colortbl}
\usepackage{pdflscape}
\usepackage{tabu}
\usepackage{threeparttable}
\usepackage{threeparttablex}
\usepackage[normalem]{ulem}
\usepackage{makecell}
\usepackage{xcolor}
\ifLuaTeX
  \usepackage{selnolig}  % disable illegal ligatures
\fi
\IfFileExists{bookmark.sty}{\usepackage{bookmark}}{\usepackage{hyperref}}
\IfFileExists{xurl.sty}{\usepackage{xurl}}{} % add URL line breaks if available
\urlstyle{same} % disable monospaced font for URLs
\hypersetup{
  pdftitle={Report Chile: data work COVID-19},
  pdfauthor={Ruggero Doino},
  hidelinks,
  pdfcreator={LaTeX via pandoc}}

\title{Report Chile: data work COVID-19}
\author{Ruggero Doino}
\date{13 September, 2022}

\begin{document}
\maketitle

{
\setcounter{tocdepth}{2}
\tableofcontents
}
\hypertarget{data-exploring}{%
\section{Data Exploring}\label{data-exploring}}

First, I download COVID-19 data from
\href{https://app.powerbi.com/view?r=eyJrIjoiNmU2NzBkNzUtYmM1Mi00NGVmLTljYWQtNTIxNTlhMTQ4ZjQ5IiwidCI6ImIwMGQ1ZjQ5LTk4YWMtNGJjNS1hMmM5LWNhZmRmNzEyMTZmMCIsImMiOjR9}{here}.
The repository contains two files: ``OC\_COVID19.csv'' (tenders related
to COVID-19 emergency) and ``OCItem\_COVID19.csv'' (list of items
related to COVID-19 emergency). From now on, I will call these datasets
tender-level and item-level emergency datasets, while the full sample of
tenders prepared by Leandro will be called chile\_compra datasets.

As I merge the chile\_compra datasets with the tender-level emergency
data, I find that only 14.38 \% of covid emergency tenders can be
identified in the chile\_compra datasets. However, the covid emergency
dataset contains

\begin{longtable}[t]{llll}
\caption{\label{tab:freq table for tag}Frequency table: Tender type and matching with covid-19}\\
\toprule
Tender Type & Matched & Unmatched & Total\\
\midrule
\cellcolor{gray!6}{COMPRA AGIL} & \cellcolor{gray!6}{0  (0\%)} & \cellcolor{gray!6}{38096 (17\%)} & \cellcolor{gray!6}{38096  (17\%)}\\
CONVENIO MARCO & 1403  (1\%) & 55764 (25\%) & 57167  (26\%)\\
\cellcolor{gray!6}{LICITACION PRIVADA} & \cellcolor{gray!6}{49  (0\%)} & \cellcolor{gray!6}{246  (0\%)} & \cellcolor{gray!6}{295   (0\%)}\\
LICITACION PUBLICA & 30261 (14\%) & 14281  (6\%) & 44542  (20\%)\\
\cellcolor{gray!6}{MICROCOMPRA} & \cellcolor{gray!6}{0  (0\%)} & \cellcolor{gray!6}{4  (0\%)} & \cellcolor{gray!6}{4   (0\%)}\\
\addlinespace
PARA INFORMAR & 0  (0\%) & 953  (0\%) & 953   (0\%)\\
\cellcolor{gray!6}{TRATO DIRECTO} & \cellcolor{gray!6}{1  (0\%)} & \cellcolor{gray!6}{79526 (36\%)} & \cellcolor{gray!6}{79527  (36\%)}\\
Total & 31714 (14\%) & 188870 (86\%) & 220584 (100\%)\\
\bottomrule
\end{longtable}

Therefore, if for example we will focus only on ``LICITACION PUBLICA'',
the matching rate would be around 67.94

\hypertarget{task-1}{%
\section{Task 1}\label{task-1}}

\hypertarget{task-2}{%
\section{Task 2}\label{task-2}}

\hypertarget{task-3}{%
\section{Task 3}\label{task-3}}

\end{document}
