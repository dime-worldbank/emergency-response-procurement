% Options for packages loaded elsewhere
\PassOptionsToPackage{unicode}{hyperref}
\PassOptionsToPackage{hyphens}{url}
%
\documentclass[
]{article}
\usepackage{amsmath,amssymb}
\usepackage{lmodern}
\usepackage{iftex}
\ifPDFTeX
  \usepackage[T1]{fontenc}
  \usepackage[utf8]{inputenc}
  \usepackage{textcomp} % provide euro and other symbols
\else % if luatex or xetex
  \usepackage{unicode-math}
  \defaultfontfeatures{Scale=MatchLowercase}
  \defaultfontfeatures[\rmfamily]{Ligatures=TeX,Scale=1}
\fi
% Use upquote if available, for straight quotes in verbatim environments
\IfFileExists{upquote.sty}{\usepackage{upquote}}{}
\IfFileExists{microtype.sty}{% use microtype if available
  \usepackage[]{microtype}
  \UseMicrotypeSet[protrusion]{basicmath} % disable protrusion for tt fonts
}{}
\makeatletter
\@ifundefined{KOMAClassName}{% if non-KOMA class
  \IfFileExists{parskip.sty}{%
    \usepackage{parskip}
  }{% else
    \setlength{\parindent}{0pt}
    \setlength{\parskip}{6pt plus 2pt minus 1pt}}
}{% if KOMA class
  \KOMAoptions{parskip=half}}
\makeatother
\usepackage{xcolor}
\usepackage[left=2cm,right=2cm,top=2cm,bottom=2cm]{geometry}
\usepackage{graphicx}
\makeatletter
\def\maxwidth{\ifdim\Gin@nat@width>\linewidth\linewidth\else\Gin@nat@width\fi}
\def\maxheight{\ifdim\Gin@nat@height>\textheight\textheight\else\Gin@nat@height\fi}
\makeatother
% Scale images if necessary, so that they will not overflow the page
% margins by default, and it is still possible to overwrite the defaults
% using explicit options in \includegraphics[width, height, ...]{}
\setkeys{Gin}{width=\maxwidth,height=\maxheight,keepaspectratio}
% Set default figure placement to htbp
\makeatletter
\def\fps@figure{htbp}
\makeatother
\setlength{\emergencystretch}{3em} % prevent overfull lines
\providecommand{\tightlist}{%
  \setlength{\itemsep}{0pt}\setlength{\parskip}{0pt}}
\setcounter{secnumdepth}{5}
\usepackage{booktabs}
\usepackage{rotating, graphicx}
\usepackage{dcolumn}
\usepackage{float}
\usepackage{enumitem}
\usepackage{booktabs}
\usepackage{longtable}
\usepackage{array}
\usepackage{multirow}
\usepackage{wrapfig}
\usepackage{float}
\usepackage{colortbl}
\usepackage{pdflscape}
\usepackage{tabu}
\usepackage{threeparttable}
\usepackage{threeparttablex}
\usepackage[normalem]{ulem}
\usepackage{makecell}
\usepackage{xcolor}
\ifLuaTeX
  \usepackage{selnolig}  % disable illegal ligatures
\fi
\IfFileExists{bookmark.sty}{\usepackage{bookmark}}{\usepackage{hyperref}}
\IfFileExists{xurl.sty}{\usepackage{xurl}}{} % add URL line breaks if available
\urlstyle{same} % disable monospaced font for URLs
\hypersetup{
  pdftitle={Report Chile: data work COVID-19},
  pdfauthor={Ruggero Doino},
  hidelinks,
  pdfcreator={LaTeX via pandoc}}

\title{Report Chile: data work COVID-19}
\author{Ruggero Doino}
\date{19 September, 2022}

\begin{document}
\maketitle

{
\setcounter{tocdepth}{2}
\tableofcontents
}
\hypertarget{data-exploring}{%
\section{Data Exploring}\label{data-exploring}}

\textbf{E-procurement}

The e-procurement portal gathers information about 1,061,341 tenders
such as ``Licitation Publica'' and ``Licitation Privada'', from 2015 to
2021. The analysis will focus only on successful tenders
(``Adjudicada''), and thus tenders that went ``Cerrada''(1.7 \%),
``Desierta'' (12 \%), ``Revocada'' (1.3 \%), ``Suspendida'' (0.011 \%)
will be excluded. As such, the final sample will account for 901,968 (85
\% of the original sample).

This dataset contains a wide set of variables summarized in the
following list:

\begin{itemize}
  \item Tender-level: description, title, date of publication/query/award/payment, type, buyer characteristics.
  \item Item-level: quantity, price, description, title, UNSPSC codes.
  \item Participant-level: bid details such as price and volumes, firms' name.
\end{itemize}

\textbf{Covid portal}

First, I download COVID-19 data from
\href{https://app.powerbi.com/view?r=eyJrIjoiNmU2NzBkNzUtYmM1Mi00NGVmLTljYWQtNTIxNTlhMTQ4ZjQ5IiwidCI6ImIwMGQ1ZjQ5LTk4YWMtNGJjNS1hMmM5LWNhZmRmNzEyMTZmMCIsImMiOjR9}{here}.
The repository contains two files: ``OC\_COVID19.csv'' (tenders related
to COVID-19 emergency) and ``OCItem\_COVID19.csv'' (list of items
related to COVID-19 emergency). From now on, I will call these datasets
tender-level and item-level emergency datasets, while the ordinary full
sample of tenders prepared by Leandro will be called chile\_compra
datasets. This dataset gathers information about 220,584 tenders with
the following set of variables:

\begin{itemize}
  \item Tender-level: description, title, no date of publication/query/award/payment, type, buyer characteristics.
  \item Item-level: quantity, price, only date of delivery, UNSPSC code, description. 
  \item Participant-level: no bid information.
\end{itemize}

\textbf{Merging two datasets}

As I merge the chile\_compra datasets with the tender-level emergency
data, I find that only 42 \% of covid emergency tenders can be
identified in the chile\_compra datasets. However, discrapencies could
be explained by several reasons:

\begin{itemize}
  \item Emergency portal covers also tenders from 2022, while our sample ranges from 2015 to 2021. Indeed, 0 (0.00 \%) are from 2022. 
  \item Emergency portal gathers information not only about "Licitation Privada" and "Licitation Publica", which are the only type of tenders present in the ordinary e-procurement portal. Indeed, the emergency portal includes 44837 (20.3 \%).
  \item Since we focus only on successful tenders from the ordinary e-procurement portal, 220584 (20 \%) unsuccessful emergency tenders will not be necessary. The table below better explores this point. 
  \item 122546 (55.6 \%) emergency tenders do not show any tender ID. 
\end{itemize}

\begin{longtable}[t]{llll}
\caption{\label{tab:freq table for tag}Frequency table: Tender type and matching with covid-19}\\
\toprule
Tender Type & Matched & Unmatched & Total\\
\midrule
\cellcolor{gray!6}{COMPRA AGIL} & \cellcolor{gray!6}{0  (0\%)} & \cellcolor{gray!6}{38096 (17\%)} & \cellcolor{gray!6}{38096  (17\%)}\\
CONVENIO MARCO & 51552 (23\%) & 5615  (3\%) & 57167  (26\%)\\
\cellcolor{gray!6}{LICITACION PRIVADA} & \cellcolor{gray!6}{49  (0\%)} & \cellcolor{gray!6}{246  (0\%)} & \cellcolor{gray!6}{295   (0\%)}\\
LICITACION PUBLICA & 40228 (18\%) & 4314  (2\%) & 44542  (20\%)\\
\cellcolor{gray!6}{MICROCOMPRA} & \cellcolor{gray!6}{0  (0\%)} & \cellcolor{gray!6}{4  (0\%)} & \cellcolor{gray!6}{4   (0\%)}\\
\addlinespace
PARA INFORMAR & 0  (0\%) & 953  (0\%) & 953   (0\%)\\
\cellcolor{gray!6}{TRATO DIRECTO} & \cellcolor{gray!6}{2  (0\%)} & \cellcolor{gray!6}{79525 (36\%)} & \cellcolor{gray!6}{79527  (36\%)}\\
Total & 91831 (42\%) & 128753 (58\%) & 220584 (100\%)\\
\bottomrule
\end{longtable}

The emergency portal can only partially help us to identify which
tenders are covid-related since the merging rate with the main sample is
only 42 \%. Therefore, I will also define tenders based on the
description of the tender.

\emph{Definition of covid-related tender} Covid-related tender is
defined as any tender that meet one of the two following conditions:

\begin{itemize}
  \item The tender has been registered in the covid emergency e-procurement portal. However, we need to keep in mind that 46 \% of tenders do not have a string such as "COVID-19", "COVID" and/or "CORONAVIRUS" in its description.  
  \item The tender has in its description a string such as "COVID-19", "COVID" and/or "CORONAVIRUS".  
\end{itemize}

data\$

\hypertarget{task-1}{%
\section{Task 1}\label{task-1}}

\begin{itemize}
  \item Product classification: We define three different categories: 
  \begin{enumerate}
    \item medical products, covid related (UNSPSC code starts with 42 + the item is part of the covid emergency data).
    \item medical products, non-covid related (UNSPSC code starts with 42 + the item is noy part of the covid emergency data)
    \item non medical products (UNSPSC code does not start with 42)
  \end{enumerate}
\end{itemize}

\hypertarget{task-2}{%
\section{Task 2}\label{task-2}}

\hypertarget{task-3}{%
\section{Task 3}\label{task-3}}

\end{document}
